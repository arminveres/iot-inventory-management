\chapter{Related Work}

% NOTE: compares, contrasts, synthesizes, and provides introspection about the available knowledge for
% a given topic or field

\section{CERTIFY} % (fold)
\label{sec:CERTIFY}

This thesis is carried out in conjunction with the CERTIFY project \cite{certifyproject2023}.

The National Institute for Standard and Technology has a few ongoing projects and white papers on security related
mitigation methods for IoT devices.

% section CERTIFY (end)


\section{DLT-based Asset-Tracking} % (fold)
\label{sec:DLT-based Asset-Tracking}

Neisse et al. (2017) analyzed how blockchain-based approaches might be used for data accountability and provenance
tracking under the then recently released GDPR legislation, highlighting challenges of scalability and considering
sharding as a method to address it. \cite{neisse2017blockchain} Further they also mentioned issues of clonability of
the tracked assets, which we can also correlate to the physical assets that are tracked inside blockchain.

% section DLT-based Asset-Tracking (end)


\section{Cybersecurity of IoT Devices} % (fold)
\label{sec:Cybersecurity of IoT Devices}


In order to maintain participation rights for only valid users/clients, Manufacturer Usage Descriptions, MUDs, are
getting more and more relevant, as also the National Institute for Standards and Technology, NIST, have been considering
their use cases. \cite{dodson2021securing}


\subsection{SRAM-based PUF Readouts} % (fold)
\label{sub:SRAM-based PUF Readouts}

For this thesis we will not be implementing a sophisticated Designated Accrediting Authority, DAA, leading to the
assumption, that it will be implemented as part of another thesis. For our use-case we will refer to simple
hardcoded authentication, with a key being provided for each device by us and trusted by us for simplicity.


Nonetheless the topic is still relevant from a cybersecurity perspective and there have been a few attempts at creating
secure keys out of SRAM readouts, such as \cite{vinagrero2023sram} or \cite{Niya_Jeffrey_Stiller_2020}.
We will also be considering using SRAM-based PUF readouts in our device configurations in order to get DIDs that are
absolutely unique for each device.
% subsection SRAM-based PUF Readouts (end)

\subsection{Device Fingerprinting} % (fold)
\label{sub:Device Fingerprinting}

For classification of device capabilities NIST has been considering the usage of MUDs, so that devices do not step out
the bounds of their official and appointed capabilities. \cite{dodson2021securing}

% subsection Device Fingerprinting (end)

% TODO: (aver) add section on VCs, compare different IdM technologies, e.g., Aries

\subsection{Verified Credentials} % (fold)
\label{sub:Verified Credentials}

\begin{table}
	\caption{Comparison of Identity Management and Verified Credentials related Software}
	\label{tab:Comparison of Identity Management and Verified Credentials related Software}
	\begin{center}
		\begin{tabular}[c]{|l|p{7cm}|l|}
			\hline
			\multicolumn{1}{|c|}{\textbf{Technology Name}} & Short Description                             &
			Main Feature                                                                                     \\
			% \multicolumn{1}{c|}{\textbf{hola}} \\ % use to add a line below or above
			\hline
			Hyperledger Aries \cite{hyperledger:wiki}      & creating, transmitting and storing verifiable
			digital credentials                            & VC issuing                                      \\
			\hline
			Hyperledger Indy \cite{hyperledger:wiki}       & Tools/Libraries for providing digital IDs
			rooted on Blockchains or DLTs                  & DID and Tools                                   \\
			\hline
		\end{tabular}
	\end{center}
\end{table}


\subsubsection{Hyperledger Aries} % (fold)
\label{sec:Hyperledger Aries}

Hyperledger Aries is a graduated project from the Hyperledger foundation that focuses on on creating, transmitting and
storing verifiable digital credentials, with its infrastructure aimed at blockchain-rooted peer-2-peer interactions.

% subsubsection Hyperledger Aries (end)
% subsection Verified Credentials (end)
% section Cybersecurity of IoT Devices (end)

\section{Blockchains} % (fold)
\label{sec:Blockchains}
\begin{table}
	\caption{Blockchains considered}
	\label{tab:Blockchains considered}
	\begin{center}
		\begin{tabular}[c]{|l|l|}
			\hline
			\multicolumn{1}{|c|}{\textbf{Blockchain Name}} & Key Characteristics  \\
			\hline
			Hyperledger Iroha \cite{hyperledger:wiki}      & Permissioned Network \\
			\hline
			Ethereum                                       & Permissionless       \\
			\hline
		\end{tabular}
	\end{center}
\end{table}

% section Blockchains (end)
